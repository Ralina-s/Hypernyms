\section{Описание тестовых данных и метрик оценки качества моделей}
\label{sec:Chapter_3} \index{Chapter_3}
\large


\subsection{Данные}

Критерием $h$ является проверка подматриц на равенство. Предположим, требуется определить, удовлетворяет ли элемент $g^s_k$ критерию h. Для этого необходимо, чтобы элементы $a_{ij}$, для всех $i,j \leq k$, были равны соответствующим элементам $a_{r_ir_j}$. Подматрица $g^s_k$ выглядит так:

\[ \bordermatrix{
& r_1 & r_2 & \dots & r_k \cr
r_1 & a_{r_1r_1} & a_{r_1r_2} & \dots & a_{r_1r_k} \cr
r_2 & a_{r_2r_1} & a'_{r_2r_2} & \dots & a_{r_2r_k} \cr
\vdots & \vdots & \vdots & \ddots & \vdots \cr
r_k & a_{r_kr_1} & a_{r_kr_2} & \dots & a_{r_kr_k} \cr}
\]

Если хотя бы одно из равенств не выполнено, это означает, что по данной частичной подстановке хотя бы одна вершина отобразилась в другую так, что структура графа изменилась. А так как частичная перестановка $g^s_k$ фиксирует $r_1, \ldots r_k$, то структура останется измененой, что означает отображение не будет являться автоморфизмом.



\subsection{Метрики качества}
