\section{Постановка задачи}
\label{sec:Chapter_1} \index{Chapter_1}
\large

Целью данной работы является реализация различных алгоритмов извлечения
гиперонимов из текстовых корпусов. Исследовать алгоритм, основанный на анализе
текстов по средством рукописных регулярных выражений, а также класс алгоритмов,
использующих разные виды нейронных сетей, с последующим обучением на размеченных
данных.
Алгоритм должен для заданного гипонима составлять упорядоченный по вероятности
список гиперонимов.\\



\textbf{ЗАДАЧИ}

\begin{enumerate}
\item Составить обзор существующих подходов к решению поставленной задачи
\item Выбор набора данных. Разделение его на обучающую и тестовую части.
\item Выбор метрик качества модели, наиболее точно отражающих главные критерии
качества. Метрики должны учитывать порядок выбранных гиперонимов, так как
основная задача - ранжирование списка.
\item Построение и тестирование модели, основанной на рукописных шаблонах
\item Исследование первой модели векторного представления слов, основанной на гипотезе
дистрибутивной семантики. PMI + SVM.
\item Построение моделей, использующих различные комбинации векторов, полученных
алгоритмом Word2Vec. Дообучение алгоритмом SGD и LambdaRank.
\item Реализация нейронной сети, разделяющей каждое слово на 2 вектора: слово в
качестве гипонима и слово в качестве гиперонима. Распараллеливание алгоритма
средствами Hadoop MapReduce.
\item Сравнение результатов, полученных всеми построенными моделями.
\end{enumerate}

