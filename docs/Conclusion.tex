\section*{Заключение}
\addcontentsline{toc}{section}{Заключение}
\label{sec:Conclusion} \index{Conclusion}
\large

В данной работе были исследованы и реализованы различные методы извлечения гиперонимов из текстовых корпусов. Рассмотрен алгоритм, основанный на анализе текстов средством рукописных регулярных выражений. Протестированы шаблоны из библиотеки \textit{hearstPatterns (Python)} на текстовом корпусе UMBC и заранее извлеченные пары ProBase. Также реализованы и проанализированы алгоритмы получения векторного представления слов: Word2Vec и DDM, с последующим их дообучением алгоритмами GBR и LambdaRank. Лучшая модель протестирована на полном наборе данных с итеративным дообучением посредством выбора определенных отрицательных примеров (NegativeExample).

Наиболее удачной оказалась связка ProBase + DDM + LambdaRank + NegativeExample. ProBase обеспечивает высокой точностью, DDM увеличивает полноту, LambdaRank и NegativeExample подстраивают полученные вектора под конкретные эталонные данные. 

Полученная модель извлекает достаточно правильные гиперонимы, относительно действительных значений слов. Увеличения точности результатов на конкретных эталонных данных, позволит составление качественных шаблонов, аналогичных ProBase, на текстовом корпусе UMBC.