\section{Исследование и построение решений задачи}
\label{sec:Chapter_4} \index{Chapter_4}
\large




\subsection{Шаблонный метод}

Для тестирования первого метода решения поставленной задачи, необходимо
было составить шаблоны поиска гипонимов и гиперонимов, аналогичных шаблонам
Марти Херст.

Были найдены уже готовые примеры шаблонов как в научных статьях, так и в
готовых программах, написанных на языке программирования. Наиболее широки
списком разнообразных шаблонов обладал класс \textit{hearstPatterns}, написанный на
языке Python. Всего в нем содержалось 48 примеров, написанных регулярными
выражениями.

Для удобной и более быстрой работы с данным классом, необходимо было
выгрузить исходный код и подправить его под свою задачу. Производилась
внутренняя лемматизация слов и приведение их к нижнему регистру, а также
удаление пунктуации предложения. В итоге, была получена функция, на вход
которой передавалось необработанное предложение, а на выходе составлялся
список найденных пар гипоним-гипероним.
Целью проверки данного метода шаблонов было извлечение всех пар, связанных
отношением is-a, из текстового корпуса UMBC средствами класса \textit{hearstPattern}s, и
последующая оценка выбранными метриками.

Как оказалось на практике, написанных шаблонов было недостаточно, для
качественного обнаружения искомых пар. В качестве примера: из 1,5 млн
предложений было обнаружено всего 8003 пары. Более того, на обработку одно
предложения всеми регулярными выражениями требовалось очень много времени.
Просмотр всего корпуса требовал более 5-ти дней. Поэтому добавление новых
рукописных шаблонов не представлялось возможным.

Результаты:

\begin{itemize}
\item MRR: 0.005
\item MAP: 0.0017
\item P@1: 0.0033
\item P@3: 0.0023
\item P@5: 0.0018
\item P@15: 0.0012
\end{itemize}

Не имея в наличии мощной техники, достаточного времени и богатого списка
шаблонов, протестировать данной метод в полном объеме не удалось.





\subsection{PPMI + SVD}

Для исследования первого метода, основанного на дистрибутивной гипотезе, каждое
предложение текстового корпуса было разбито на контексты с шириной окна в 5 слов.
Составлена разреженная матрица частоты встречаемости пары (слово, контекст). Затем
полученная матрица была пересчитана в PPMI матрицу по формулам, описанным в пункте
2.1.

Размер матрицы оказался слишком большим, чтобы было возможно применить алгоритм
снижения размерности SVD. Были опробованы готовые реализации модели на языках
Python (классы numpy.linalg.svd и scipy.sparse.linalg.svds) и Matlab (svd), но ни одна из них не смогла преобразовать PPMI матрицу.

Таким образом, метод PPMI + SVD, получения векторного представления слов,
протестировать на существующий данных не удалось.





\subsection{WORD2VEC}

\subsubsection{Модель GOOGLENEWS}

Следующим исследуемым методом получения векторного представления слов был
Word2Vec.
Существует множество готовых обученных моделей, опубликованных в открытом доступе.
Для тестирования решения данной задачи была применена модель, имеющая архитектуру
CBOW и обученная на корпусе GoogleNews, размером в 3 миллиона слов. Основные
параметры: размерность вектора 300, размер окна 5.
Алгоритм Word2Vec основывается на информации о контекстах, в которых употреблялось
слово в текстовом корпусе, поэтому получение вектора возможно только для тех слов,
которые встречались в корпусе хотя бы 1 раз (в общем случае, не менее N раз, где порог N
является гиперпараметром). Значит, если гипоним, для которого необходимо найти все
гиперонимы, не встречался, в обучающем корпусе, то построенная модель не сможет
подобрать для него требуемый список.
Из 1500 гипонимов, находящихся в выбранном корпусе данных, построенная модель,
смогла предоставить вектора только для 77%. Для оставшихся слов, необходимо было
применять другую модель.
Для получиния итогового списка гиперонимов к каждому гипониму, было применено 2
различных алгоритма: GBR (Gradient Boosting for regression) и LambdaRank.
Алгоритм GBR можно применить как pointwise алгоритм ранжирования. В то время как
LambdaRank является представителем pairwise подхода. Нельзя заранее сказать, какой из
двух подходов будет работать лучше, поэтому для тестирования были использованы оба.

\paragraph{Подготовка обучающего множества}

Так как необходимо было исследовать множество алгоритмов с различными параметрами,
для экономии скорости и минимальной потери точности, извлечение гиперонимов
происходило не из полного словаря, содержащего $\approx 220$ тыс слов. Для каждого гипонима
составлялся свой словарь по следующему алгоритму:

\begin{enumerate}
\item Добавлялись все слова из эталонного списка гиперонимов. Среднее количество
гиперонимов для каждого гипонима составляло 5 слов

\item Каждому слову из списка присваивалось значение, отражающее его позицию. Эти
величины служат целевым признаком для предсказания. Для алгоритма
LambdaRank чаще всего используются значения 0,1,2 и 3, поэтому для эталонных
гиперонимов были выбраны значения 1,2 и 3. Список делился на три части, если
слово находилось в первой из них, то ему присваивалась величина 3, если во
второй, то 2, оставшейся последней части - 1.

\item Далее в случайном порядке добавлялось к полученному списку дополнительно 500
слов, играющих роль негативных примеров. Каждое такое слово имело целевое
значение 0.
\end{enumerate}

В качестве признаков для обучения моделей были протестированны следующие
комбинации векторов гипонима и гиперонима:

\begin{enumerate}
\item Разность векторов
$Diff: <u - v>$

\item Конкатенация векторов + евклидово расстояние степени 1
$||u - v||_1 = \sum_{i=1}^{|u_i - v_i|}$
$Dist1: <u, v, ||u - v||_1>$

\item Конкатенация векторов + евклидово расстояние степени 2
$||u - v||_2 = \sqrt{\sum_{i=1}^{|(ui - vi)^2|}}$
$Dist2: <u, v, || u - v ||_2>$

\item Конкатенация векторов + косинусное расстояние между ними
$Cos: <u, v, cos(u, v)>$
\end{enumerate}

Полученный новый набор данных разделялся на обучающую и тестовую выборку в
отношении 2 : 1. Так как не для всех гипонимов построены вектора, то тестирование и
обучение происходило только на 77% данных.

\paragraph{Тестирование}

Получены следующие результаты:

КАРТИНКА

Лучший результат среди всех опробованных моделей с векторами Word2Vec - GoogleNews,
показал алгоритм LambdaRank с основными параметрами: шаг обучения = 0.1, кол-во
деревьев 100, доля выборки на каждом шаге обучения (subsample) = 0.8, максимальная
глубина 5. Для метрики MRR, более успешными было представление векторов
комбинацией Dist1, а для MAP - Dist2.

\subsubsection{Обучение собственное модели WORD2VEC}

Далее, был обучен алгоритм Word2Vec на текстовом корпусе UBMC. Целью такого
исследования было увеличение доли гипонимов, для которых возможно построить вектор,
и возможность настроить главные гиперпараметры.
Из опробованных комбинаций параметров, лучший результат показала модель Skip-gramm,
с ширеной окна 7 и размером вектора 300.
Так как среди гипонимов встречались не только слова, но и словочетания из 2-3 слов, не
удалость построить вектора для них в сех. Доля с 77% увеличилась до 82%.

\paragraph{Тестирование}

Были применены все те подходы которые использовались для эксперимента с моделью
Word2Vec - GoogleNews.
Получены следующие результаты:
Лучшей также оказалась модель LambdaRank с теми же параметрами
LambdaRank
GBR

КАРТИНКА



\subsection{DYNAMIC DISTANCE-MARGIN MODEL}

Для применения данного алгоритма, необходимо иметь набор триплетов $(u, v, q)$, где $u$ - гипоним, $v$ - гипероним, а $q$ - сколько раз пара гипоним-гипероним $(u, v)$ встретилась в текстовом корпусе. В качестве такого набора данных был взят ProBase, описанный в главе
шаблонных методов.

ProBase был получен на основе очень больших наборов текстовых корпусов, поэтому
значения q могли достигать величины в 35000. Алгоритм DDM (Dynamic distance-margin),
учитвает и обучает каждую пару столько раз, сколько она встречалась. Таким образом,
некоторые пары могли учитываться 35 тысяч раз за одну эпоху, в то время, как
большинство других имели значение $q < 20$. В таком случае модель могла практически не
обучить большинсво векторов. Чтобы устранить такой большой разрыв значение q было
изменено по формуле $\sqrt{q}$ степени 1.85. Степень корня подбиралась так, чтобы
максимальное значение не было слишком большим или слишком маленьким.

Для каждого триплета $x$ подбирался негативный триплет $x'$, где был заменен либо
гипероним, либо гипоним на случайный. Чтобы детерменировать данный выбор и
невносить различия в обучение, было решено для каждой пары подбирать сразу 2
негативного триплета - негативный гипоним и негативный гипероним.

Нейронная сеть, используемая в данной модели, обучает входные вектора. Поэтому, было
решено, вручную расчитать градиенты, для изменения этих векторов (метод обратного
распространения ошибки). Подробное их вычисление предоставлено в приложении 3.

После рассчетов, получился следующий алгоритм изменения векторов на каждой эпохе:

for x = (u, v, q) in X: // для каждой пары из ProBase
for i in [1 . . q]: // для каждой пары негативных примеров
x’i = (u, v’i, q’i) // негативный пример гиперонима
if f(x) + log(q) < f(x’i) + log(q’i):
u += (u - v) / || u - v ||2
v -= (u - v) / || u - v ||2
u -= (u - v’) / || u - v’ ||2
v’ += (u - v’) / || u - v’ ||2
x’i = (u’i, v, q’i) // негативный пример гипонима
if f(x) + log(q) < f(x’i) + log(q’i):
u += (u - v) / || u - v ||2
v -= (u - v) / || u - v ||2
u’ -= (u’ - v) / || u’ - v ||2
v += (u’ - v) / || u’ - v ||2

Для одной эпохи необходимо было рассчитать расстояния и градиент для 10000000
(TODO) (удвоенная сумма всех $q$).
Был реализован алгоритм на языке Python, но как и следовало ожидать, для его обучения
не было достаточно памяти и разумного времени.
После исследования данных ProBase, было установлено, что не со всеми словами из этого
набора, связаны необходимые для нашей задачи гипонимы. Была выделена отдельная
компонента связности, составляющая 1 / 15 часть всего набора пар ProBase. То есть
обучение 14 / 15 всех пар векторов никак не влияло на улучшение результата
поставленной задачи.

После уменьшение данных в 15 раз была проведена еще одна попытка запуска алгоритма,
написанного на языке Python. Уменьшение данных все равно не позволило обучить
алгоритм, что привело к поиску других решений реализации данной модели.

\subsubsection{Распараллеливание алгоритма средствами HADOOP}

Одним из используемых современных решений работы с большим объемом данных
является парадигма параллельных вычислений Hadoop MapReduce. Хорошо иллюстрирует
данный подход схема, расположенная ниже.

КАРТИНКА

На стадии Map параллельно считываются данные из независимых блоков, на которые
разбит исходный входной файл. Происходит их преобразование в пары (Ключ, Значение).
Затем идет стадия Shuffle: вывод функции Map распределяется по «корзинам», в
зависимости от значения ключа. Все пары (Ключ, Значение), имеющие одинаковый ключ,
попадают в одну корзину и отдаются определенному редьюсеру. Таким образом, данные
из разных частей входного файла могут собраться вместе при обработке на стадии
Reduce. Такая архитектура позволяет быстрое распараллеливание и не загружает
оперативную память, что необходимо при решении поставленной задачи.

Краткая реализация алгоритма DDM в парадигме MapReduce:

\begin{enumerate}
\item Считываются значения вектаров для каждого слова. Для первой эпохи значения
заполняются случайными величинами в диапазоне $[-0.1, 0.1]$.

\item На стадии Map происходит параллельное считывание пар $(u, v, q)$.

\item Формируется пара (Ключ, Значение) для её обработки на стадии Reduce. Ключ = $u$,
Значение = $v$. Отправляются на Reduce.

\item Для каждой такой пары генерируется $q$ негативных примеров гипонимов $u'$ и $q$
негативных примеров гиперонимов $v'$.

\item Для каждой пары $(u, v')$ ( аналогично для $(u', v)$) рассчитывается разница расстояния между $u$ и $v'$ и расстояния между $u$, $v$. Если $f(x) + \log(q) < f(x') + \log(q')$, то необходимо изменить вектора $u, v, v'$ - переход к пункту 4. Иначе просматривается следующая
пара.

\item Рассчитываются антиградиенты $du, dv, dv'$ и умножаются на шаг обучения

\item Формируются пары (Ключ, Значение) для их обработки на стадии Reduce. Ключ
соответствует самому вектору, а значение - изменению. Получаются пары $(u, du)$, $(v,
dv)$, $(v’, dv’)$. Отправляются на Reduce.

\item На стадии Reduce для каждого вектора $u$ (аналогично $v$), собирается список всех
его изменений, полученных из пункта 6 и изначальное значение вектора из пункта 1.

\item Высчитывается новое значение вектора и сохраняется в файле, для его считывания
следующий эпохой.
\end{enumerate}


\subsubsection{Тестирование}

При распараллеливании данного алгоритма на 24 кластера, получилось, что средняя
продолжительность вычисления одной эпохи составляет 50 секунд.
Параметры сети были установлены следующими: длина векора 100, шаг обучения 0.1, кол-
во эпох 100.
Также, как и при исследовании метода Word2Vec были протестированны алгоритмы GBR и
LambdaRank с видами представления векторов: $Diff, Dist1, Dist2, Cos$.
Полученные результаты:

[ТАБЛИЧКА]